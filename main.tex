\documentclass[zihao = -4,cn]{oucart}

\title{论文标题}
\entitle{The Title of the Thesis}
\author{作者名}
\advisor{指导教师名}
\department{学院名}{专业年级}

\cnabstractkeywords{
 这是中文摘要。 
}{
中文; 关键字
}
\enabstractkeywords{
  This is an English abstract. 
}{
  English; Abstract
}

\begin{document}
	
\pagestyle{fancy}
\renewcommand{\headrulewidth}{1pt}
\fancyhead[C]{武汉纺织大学2024届毕业设计论文}

\makecover

% \makesignature

\makeabstract


\begin{spacing}{1.75}
\tableofcontents
\end{spacing}

\newpage
\pagenumbering{arabic}
\setcounter{page}{1} 
% 正文内容
% 建议使用 \input{<文件名>} 指令引用其他文件
\section{示例章节}
\subsection{示例章节}
\subsubsection{示例章节}
武汉纺织大学位于中国湖北省武汉市,现有南湖校区(老校区,位于鲁巷纺织路一号)、阳光校区(主校区,2004年开始招生)、东湖校区(湖北财经高等专科学校)三大校区,占地面积2400余亩,建筑面积70多万平方米。学科覆盖了工学、理学、哲学、文学、管理学、经济学、法学和教育学八个学科门类,以纺织服装为特色,42个本科专业,现在有校本科生、研究生一共14000余人\cite{wiki:xxx}. 
\section{测试章节}
测试图片:
\begin{figure}[!htbp]
    \centering
    \includegraphics[width=5cm]{figures/wtulogo}
    \caption{武汉纺织大学}
    \label{fig:ouc1}
\end{figure}

测试表格:
\begin{table}[!htbp]
\centering
\caption{一个基本的三线表}
\begin{minipage}[t]{350pt}
\begin{tabular*}{350pt}{@{\extracolsep{\fill}}ccc}
\toprule
第一列 & 第二列 & 第三列 \\
\midrule
文字 & English & $\alpha^*$ \\
文字 & English & $\beta$ \\
文字 & English & $\gamma$\\
\bottomrule
\end{tabular*}
\footnotesize
数据来源:相关的数据来源。 \\
$*$:表中需要解释的内容
\end{minipage}
\end{table}

测试公式:
\begin{equation}
 \lim_{x\to 0}{\frac{e^x-1}{2x}}
 \overset{\left[\frac{0}{0}\right]}{\underset{\mathrm{H}}{=}}
 \lim_{x\to 0}{\frac{e^x}{2}}={\frac{1}{2}}
\end{equation}
\newpage

%\bibliographystyle{unsrt}
\bibliography{main}

\newpage


\begin{center}
\zihao{3} \textbf{附\ \ 录} \\
\end{center}
\lstinputlisting[
style       =   Python,
caption     =   {\bf ff.py},
label       =   {ff.py}
]{codes/code1.py}

\newpage
\begin{center}
	\zihao{3} \textbf{致\ \ 谢} \\
\end{center}

\end{document}